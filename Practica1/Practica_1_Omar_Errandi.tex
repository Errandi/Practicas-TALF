\documentclass{article}
%Gummi|065|=)
\title{\textbf{Practica 1}}
\author{Omar Errandi}
\date{}
\begin{document}

\maketitle

\section{Find theFind  the  power  set
R$^3$ of R = {(1,1),(1,2),(2,3),(3,4)}.   Check  your  answer with the script powerrelation.m and write a LATEX document with the solution step by step.}

	Como bien sabemos para calcular la potencia de un conjunto debemos de conocer las potencias anteriores. En este caso debemos de conocer la potencia 2 de la relacion (R$^2$).

Checkeamos los apuntes y buscamos la definición de potencia de una  relacion:

R$^n$ = $\Biggl\{$$\begin{array}{ll}
		 R      & \mathrm{si\ } n = 0 \\
		 (\{ (a,b) : \exists x \in A, (a,x)\in R^{n-1}  \wedge (x,b)\in R \}) & \mathrm{si\ }  n > 0 \\
		 
	       \end{array}$



Procedemos a calcular la potencia dos de la relacion fijándonos en la formula anterior:

R$^2$ = \{ (1,1), (1,2), (1,3), (1,4) \}

Ya tenemos calculada la segunda potencia, por lo que ya podemos calcular la tercera (R$^3$):

R$^3$ = \{ (1,1), (1,2), (1,3), (1,4) \}

Ya tenemos la solucion. Podemos comprobar con Octave si es correcto o no mediante los siguientes comandos:


\emph powerrelation ({['1','1'], ['1','2'],['2','3'],['3','4']}, 2)

\emph powerrelation ({['1','1'], ['1','2'],['2','3'],['3','4']}, 3) 












\section{Within the folder “files”,find a TEX file in whose content appears the string  $\backslash$usepackage \{amsthm,amsmath\}.   Note: use grep and  escape  the  special characters with \ .  Complete the proof and answer the question.
}

Primero de todo abrimos la consola (bash de linux) y con el comando "ls" o "dir" vemos el directorio en el que estamos. 
Una vez visto, nos dirijimos a la carpeta descargada "Practica1.tar.gz" descomprimida (con el comando "cd"). 

Situados ya  en la carpeta Practica1 usamos el comando grep, que nos  permite filtrar y  encontrar las coincidencias. A continuacion usamos el string  que nos proporcionan en el  enunciado que hay que buscar ($\backslash$usepackage\{amsthm, amsmath\}).
Y para finalizar el directorio a analizar desde el directorio en el que estamos "files/*"(el asterisco final es para buscar en todos los documentos de la carpeta).
Con lo que nos queda el comando:



\emph grep " $\backslash$usepackage\{amsthm, amsmath\}" files/*
\\

Una vez ejecutado obtenemos el resultado:




\emph files/mainP.tex:$\backslash$usepackage\{amsthm, amsmath\}


El resultado es el archivo que contiene dicho string, que es el archivo "mainP.tex".






\end{document}
